\documentclass{article}
\usepackage{tikz}
\usetikzlibrary{calc, lindenmayersystems,shapes,decorations,decorations.shapes}
\begin{document}

\def\pointlistleft{}
\def\pointlistright{}
\pgfmathsetseed{\year * 6}
\makeatletter
\pgfdeclarelindenmayersystem{Christmas tree}{
    \symbol{C}{\pgfgettransform{\t} \expandafter\g@addto@macro\expandafter\pointlistleft\expandafter{\expandafter{\t}}}
    \symbol{c}{\pgfgettransform{\t} \expandafter\g@addto@macro\expandafter\pointlistright\expandafter{\expandafter{\t}}}
    \rule{S -> [+++G][---g]TS}
    \rule{G -> +H[-G]CL}
    \rule{H -> -G[+H]CL}
    \rule{g -> +h[-g]cL}
    \rule{h -> -g[+h]cL}
    \rule{T -> TL}
    \rule{L -> [-FFF][+FFF]F}
}
\makeatother

\begin{tikzpicture}[rotate=90]
\draw [color=brown!50!black,line width=2.5pt] (0,0) -- (170pt,0);
\draw [color=green!50!black,l-system={Christmas tree,step=4pt,angle=16,axiom=LLLLLLSLFFF,order=10,randomize angle percent=20}] lindenmayer system -- cycle;

\pgfmathdeclarerandomlist{pointsleft}{\pointlistleft}
\pgfmathdeclarerandomlist{pointsright}{\pointlistright}
\pgfmathdeclarerandomlist{colors}{{red}{blue}{yellow}}

\foreach \i in {0,1,...,5}
{
    \pgfmathrandomitem{\c}{pointsleft}
    \pgfsettransform{\c}
    \pgfgettransformentries{\a}{\b}{\c}{\d}{\xx}{\yy}
    \pgfmathrandomitem{\c}{pointsright}
    \pgfsettransform{\c}
    \pgfgettransformentries{\a}{\b}{\c}{\d}{\XX}{\YY}
    \pgftransformreset

    \pgfmathsetmacro\calcy{min(\yy,\YY)-max((abs(\yy-\YY))/3,25pt)}

    \draw[draw=orange!50!black, fill=orange!50, decorate, decoration={shape backgrounds, shape=star, shape sep=3pt, shape size=4pt}, star points=5] (\xx,\yy) .. controls (\xx,\calcy pt) and (\XX,\calcy pt) .. (\XX,\YY);
}

\foreach \i in {0,1,...,15}
{
    \pgfmathrandomitem{\c}{pointsleft}
    \pgfsettransform{\c}
    \pgftransformresetnontranslations
    \draw[color=black] (0,0) -- (0,-4pt);
    \pgfmathrandomitem{\c}{colors}
    \shadedraw[ball color=\c] (0,-8pt) circle [radius=4pt];
}

\foreach \i in {0,1,...,15}
{
    \pgfmathrandomitem{\c}{pointsright}
    \pgfsettransform{\c}
    \pgftransformresetnontranslations
    \draw[color=black] (0,0) -- (0,-4pt);
    \pgfmathrandomitem{\c}{colors}
    \shadedraw[ball color=\c] (0,-8pt) circle [radius=4pt];
}

\end{tikzpicture}
\end{document}
